7-ago-2019
\section{Curvas regulares}
\begin{proposition}
  Sea $f:A \subset \mathbb{R}^n \to \mathbb{R}^,$ una función que sea continua e
  inyectiva. Si $A$ es compacto, entonces $f$ es un homeomorfismo sobre su
  imagen, es decir, $f: A \to f(A)$ es una biyección continua cuya inversa es
  continua.
\end{proposition}
\begin{proof}
\end{proof}
\begin{definition}
  Una función $f$ es localmente inyectiva en $X$ si $\forall \, x \in X \quad
  \exists \, \delta > 0 $ tal que $f:E(x,\delta) \to \mathbb{R}^n$ es inyectiva
  sobre su imagen. Se define de forma análoga, reemplazando ``inyectiva'' por ``un
  homeomorfismo'' el que una función sea localmente un homeomorfismo.
\end{definition}
\begin{proposition}
  Sea $f: I \to \mathbb{R}^n$ continua, localmente inyectiva. Entonces $f$ es
  localmente un homeomorfismo. 
\end{proposition}
La demostración es consecuencia de la proposición 1.
\begin{proposition}
  Sea $f:D \subset \mathbb{R} \to \mathbb{R}^n$, con $n>1$. Si $f$ es un
  homeomorfismo sobre su imagen, entonces $\mathring{f(D)} = \emptyset$.
\end{proposition}
\begin{proof}
  Procediendo por contradicción suponga que $\mathring{f(D)} \neq \emptyset$,
  así, $\exists \, u \in f(D)$ tal que $\exists \, \delta > 0$ con
  $E(u,\delta)\subset f(D)$. Sea $G = E(u,\delta)$ y sea $J = f^{-1}(G)$. $G$ es
  conexo en $\mathbb{R}^n$. Al ser $f^{-1}$ continua, $J$ es conexo en
  $\mathbb{R}$, además $\mathring{J} \neq \emptyset$. Considere a $x\in
  \mathring{J}$ y a $v = f(x)$. Veamos ahora que la imagen inversa de dicho
  conjunto excluyendo a $x$, es decir, $f^{-1}(G\setminus\{v\}) = J \setminus
  \{x\}$ es conexo en $\mathbb{R}$. Sin embargo, $x \in \mathring{J}$, así $J
  \setminus \{x\}$ no es conexo, llegando así a una contradicción que parte de
  suponer que $\mathring{f(D)}\neq \emptyset$.
\end{proof}
\begin{lemma}
  Sean $A,B$ subconjuntos de $\mathbb{R}^n$, $A$ cerrado. Si $\mathring{A} =
  \mathring{B} = \emptyset$ entonces $\mathring{(A\cup B)} = \emptyset$.
\end{lemma}
\begin{proof}
  Suponga que 
\end{proof}
9-ago-2019
\begin{proposition}
  Sea $f: I \to \mathbb{R}^n$ una función continua localmente inyectiva,
  entonces $int(f(I)) = \emptyset$. 
\end{proposition}
\begin{proof}
  La proposición 2 implica que $f$ es localmente un homeomorfismo, ees decir,
  para cada $c \in I, \exists \delta_x > 0$ tal que $f$ restringido a  $E(x,
  \delta_x$ es un homeomorfismo sobre su imagen. La proposición 3 implica que
  $int[f(E(x,\delta_x)] = \emptyset$. Para cada $x \in I$ sea $U_x = E (x,
  \frac{\delta_x}{2}$  Así el conjunto de los $U_x$ forma una cubierta abierta de
  $I$. Como $I$ es compacto entonces existe una subcubierta finita en la familia
  $\{U_x\}$. Considerela $\{U_{x_i}\} i \leq k$. Con  $I \subset \bigcup_{i=1}^k
  U_{x_i} \subset \overline{\bigcup_{i=1}^k U_{x_i}}$.
  Así 
  \begin{align*}
    I &= I \cap \bigcup_{i=1}^k \overline{U_{x_i}} \\
    f(I) &= f(I \cap \bigcup_{i=1}^k \overline{U_{x_i}}) \\
         &=f( \bigcup_{i=1}^k I \cap \overline{U_{x_i}}) \\
         &=\bigcup_{i=1}^kf(  I \cap \overline{U_{x_i}})
  \end{align*}
  Observe que: 
  \begin{align*}
    I \cap \overline{U_{x_i}} &\subset I\cap E(x,\delta_x) \\
    f(I \cap \overline{U_{x_i}}) &\subset f(I\cap E(x,\delta_x))
  \end{align*}
  Veamos que cada $f(I \cap \overline{U_{x_i}})$ es compacto en $\mathbb{R}^n$ y cada
  $int[f(I \cap \overline{U_{x_i}})]$
\end{proof}
\begin{definition}
  Sea $I$ un intervalo cerrado, se dice que una función $f: I \to \mathbb{R}^n$
  siendo $n >1$ es una curva regular si se cumplen dos condiciones:
  \begin{enumerate}
    \item $f'$ está definida y es continua en $I$.
    \item $f'(t) \neq 0, \quad \forall t \in I$  
  \end{enumerate}
\end{definition}
\begin{proposition}
  Sea $f:I \to \mathbb{R}^n$ una curva regular, entonces $f(I)$ no tiene puntos
  interiores.
\end{proposition}
\begin{proof}
  De acuerdo con la proposición 4, basta probar que $f$ es continua y localmente
  inyectiva. $f$ es continua, por ser deribable. Para probar que $f$ es
  localmente inyectiva, tomemos dos $t_0,t_1$, distintos arbitrarios en $I$.
  Asuma $t_0<t_1$
  Veamos que podemos escribir
  \[
    f(t_0) - f(t_1) = (f_1(t_0) - f_1(t_1),\ldots,f_n(t_0) - f_n(t_1))
  \]   
  Dado $i \in \{1, \ldots, n\}$, por el teorema del valor medio existe $\zeta_i
  \in (t_0, t_1)$ tal que 
  \[
    f_i(t_0)- f_i(t_1) = f_i'(\zeta_i)(t_1-t_0). 
  \]
  Sea $u=(f'_1(z_1), \ldots, f'_n(z_n))$. Entonces $f(t_0)-f(t_1) = u(t_0-t_1)$ 
  Veamos que para cada $\epsilon > 0, \exists \delta >0$ que no depende ni de
  $t_0$ ni de $t_1$ tal que si $|t_0-t_1|<\delta$, entonces $u \in E(f'(t_0),
  \epsilon)$. Podemos mostrar éste hecho de la siguiente forma:
  \[
    ||f'(t_0) - u || = \big[ \sum_{i=1}^n [f'_i(t_0)-f'_(\zeta_i)] \big]
  \]
  Por ser $f$ una curva regular $f'_i$ es continua en $I$ y por ser $I$ un
  compacto $f'_i$ es uniformemente continua. Así dado $\epsilon  > 0  \, \exists
  \delta_i > 0$ tal que $|t_0 - t_1| < \delta_i $ implica que $||f_i(t_0) -f_i(t_1)||
  < \epsilon$. Sea $\delta = \min\{\delta_i\}$. Veamos que $\delta$ solamente
  depende de $\epsilon$. Si $|t_0 - t_1|< \delta$ se tiene $|t_0 -
  \zeta_i|<\delta$ y así $|f'_i(t_0)-f'_i(\zeta_i)| < \frac{\epsilon}{\sqrt{n}}
  \quad \quad \forall i \in \{1,\ldots,n\}$. Así vemos que:
  \begin{align*}
    ||f'(t_0)- u||^2 < \sum_{i=1}^n \frac{\epsilon^2}{n} &= \epsilon^n \\
    ||f'(t_0)- u|| &= \epsilon
  \end{align*}
  Con lo anterior vemos que el hecho enunciado es cierto. 
  Por ser $f$ una curva regular $f' (t_0) \neq 0$.  Así existe $\epsilon> 0$ tal
  que  $0 \notin E(f'(t_0,t_1))$ Así 
\end{proof}
\begin{remark}
  Se ha probado que toda curva regular es localmente inyectiva. 
\end{remark}
\begin{remark}
  Dada una curva regular $f: I \to \mathbb{R}$  $f(I)$ se denomina la traza de
  la curva.
\end{remark}
12-agosto-2019
Considere una función $f: [a,b] \to \mathbb{R}^m$ Sea $mathfrak{P}$ el conjunto
de particiones de $[a,b]$ y sea $P \in \mathfrak{P}; P = \{t_0, \ldots. t_k\}$
con $a =t_0 < t_1 < \ldots < t_k = b$. $P$ determina la poligonal cuyos
segmentos son $\overline{f(t_i)f(t_{i+1})}$ definido como $\{f(t_i) + \alpha
[f(t_{i+1}-f(t_i) ])\}$ y se define $L_P$ como la longitud de tal poligonal de
la siguiente forma:
\[
  L_P = \sum_{i=0}^{k-1} || f(t_i)-f(t_{i+1}) ||
\]
Si el conjunto $\{L_P | P \in \mathfrak{P}\}$ es acotado superiormento, entonces
entonces se dice que el supremo de tal conjunto denotado por $L$  es la longitud
de $f$. En caso de que ésto ocurra, entonces diremos que $f$ es rectificable
\begin{lemma}
  Sea $f:[a,b] \to \mathbb{R}^n$ una función, sean $P_1, P_2 \in \mathfrak{P}$.
  Si $P_1$ es un refinamiento de $P_2$, entonces $L_{P_2} \leq L_{P_1}$. 
\end{lemma}
\begin{proof}
  Sea $m$ el número de elementos de $P_1 \setminus P_2$. Procediendo por
  inducción sobre $m$. Tenemos que si $m = 1$ $P_1 = P_2 \cup \{t\}$. Veamos que
  si $P_2 = \{t_0, \ldots, t_k\}$
  (ahorita completar)
\end{proof}
\begin{lemma}
  Sea $g: [a,b] \to \mathbb{R}^n$ una función continua. Entonces $\quad \forall
  \epsilon > 0, \exists \delta > 0$ tal que 
  \begin{align*}
    \mbox{si  } x,y_1,\ldots,y_n \in [a,b] \, \land \, |x-y_i| &< \delta,  \\
    \mbox{entonces  } ||g(x) - (g_1(y_1),\ldots, g_n(y_n))|| &< \epsilon
  \end{align*}
\end{lemma}
\begin{proof}
  Por ser $g$ continua, sus componentes $g_1, \ldots, g_n$ son también continuas
  y por lo tanto uniformemente continuas dado que $[a,b]$ es un compacto.
  Sea $\epsilon > 0$; sea $\epsilon' = \frac{\epsilon}{\sqrt{n}}$, entonces para
  cada $i \in \{1,\ldots,n\}$ existe $\delta_i>0$ tal que $|g_i(x) -g(y_i)|<
  \epsilon'$ si $|x-y_i| < \delta_i$. Sea $\delta = \min\{\delta_i\}$, entonces
  se tiene que:
  \begin{align*}
    ||g(x) - (g_1(y_1), \ldots, g_n(y_n))|| &= \big[ \sum_{i=1}^n (g_i(x) - g_i(y_i))^2\big] \\
                                            &< \big[\frac{\epsilon^2}{n}\big]^{1/2} = \epsilon
  \end{align*}
\end{proof}
Del teorema 7.4.5 de la introducción al análisis real de Bartle. Se
tiene que
\begin{proposition*}
  $\quad \forall \epsilon > 0, \, \exists \delta > 0$ tal que si $P \in \mathfrak{P}$
  y $||P|| < \delta$ entonces:
  \[
    \big| \int_a^b g - S_P \big| < \epsilon
  \]
  Para cada suma de Riemman $S_P$
\end{proposition*}
\begin{proposition}
  Sea $f: [a,b] \to \mathbb{R}^n$. Si $f'$ está definida y es continua en
  $[a,b]$, entonces $f$ es rectificable y 
  \[
    L = \int_a^b ||f'||
  \]
\end{proposition}
\begin{proof}
  Veamos que $\int_a^b ||f'||$ está definida al ser $||f'||$ contínua. 
  Se verificará que $f$ es rectificable. sea $P= \{t_0, \ldots, t_k \}$ una
  partición de $[a,b]$.
  Veamos que 
  \begin{align*}
    L_P &= \sum_{i=1}^k ||f(t_i)- f(t_{i-1})|| \\
        &= \sum_{i=1}^k (t_i - t_{i-1})\big[\sum_{j=1}^n f_j'(c_i^j)^2 \big]^{\frac{1}{2}}
  \end{align*}
  Cada $f_j'$ es acotada en $[a,b]$, por ser continua, así $\exists M_j$ tal que 
  \[
    f_j'(x) \leq M_j \quad \forall x \in [a,b] 
  \]
  Así
  \begin{align*}
    L_P &\leq \sum_{i=1}^k(t_i-t_{i-1})\sqrt{M_1^2 + \ldots + M_n^2}  \\
        &= (b-a) \sqrt{M_1^2 + \ldots + M_n^2}
  \end{align*}
  lo cual muestra que $\{L_P | P \in \mathfrak{P} \}$ es acotado superiormente.
  Así $f$ es rectificable.
  Verifiquemos que $|L - \int_a^b ||f'||\,| < \epsilon \quad \quad \forall \epsilon>0
  $.
  Veamos que 
  \[
    |\int_a^b ||f'|| - L| = \int_a^b ||f'|| - S_P + S_P - L_P + L_P -
    L|
  \]
  Por la desigualdad del triángulo:
  \[
    |\int_a^b ||f'|| - L| \leq |\int_a^b ||f'||\, - S_P| + |S_P - L_P| + |L_P -
    L|
  \]
  Así basta probar que cada uno de los términos de la suma anterior es menor que
  $\frac{\epsilon}{3}$
  Por la referencia mencionada, existe $\delta > 0$ tal que  si $||P_1|| < \delta$
  entonces:
  \[
    |\int_a^b ||f'||\, - S_P| < \frac{\epsilon}{3}
  \]
  Por otra parte como $L = \sup\{L_P\}$ entonces existe $P_2$ tal que:
  \[
    L_{P_2} -L < \frac{\epsilon}{3}
  \]
  Ahora dada una partición $P' = \{t_0, \ldots, t_k\}$ entonces
  \begin{align*}
    |S_{P'} - L_{P'} | &= \big| \sum_{i=1}^k(t_i - t_{i-1}) ||f'||(\zeta_i) -
    \sum_{i=1}^k||f(t_i) - f(t_{i-1})||\,\big|  \\
                       &= \big|\sum_{i=1}^k(t_i- t_{i-1})\big[ ||f'||(\zeta_i) -
                       ||f'_1(c_i^j), \ldots, f'_n(c_n^j)||\big]\,\big|\\
                       &\leq \sum_{i=1}^k(t_i- t_{i-1}) ||f'(\zeta_i) -
                       (f'_1(c_i^j), \ldots, f'_n(c_n^j))||
                       \\
  \end{align*}
  Por el lema 2 , $\exists \, \delta>0$ tal que $||f'(\zeta_i) - (f'_1(c_i^j), 
  \ldots, f'_n(c_n^j))||<\frac{\epsilon}{3(b-a)}$ para $||P_3|| < \delta_3$
  \begin{align*}
    \sum_{i=1}^k(t_i- t_{i-1}) ||f'(\zeta_i) -(f'_1(c_i^j), \ldots, f'_n(c_n^j))|| 
  &< \sum_{i=1}^k(t_i- t_{i-1})\frac{\epsilon}{3(b-a)} \\
  &= \frac{\epsilon}{3}
  \end{align*}
  Así si $P$ es un refinamiento de $P_1$, $P_2$, $P_3$, se verifica en cada caso
  la desigualdad. Sea $P = P_1 \cup P_2 \cup P_3$, se tiene entonces que:
  \begin{align*}
    |\int_a^b ||f'|| - L| &\leq |\int_a^b ||f'||\, - S_P| + |S_P - L_P| + |L_P -
    L| \\
                          &< \frac{\epsilon}{3} + \frac{\epsilon}{3} +
                          \frac{\epsilon}{3} = \epsilon
  \end{align*}
  Así 
  \[
    \int_a^b ||f'|| = L
  \]
\end{proof}
14-ago-2019
\begin{example}
  Considere la función:
  \begin{align*}
    f:[2\pi] &\to \mathbb{R}^2 \\
    t &\mapsto (\cos t, \sin t)
  \end{align*}
  Sea $S' = \{(x,y) \in \mathbb{R}^2 \, | \, x^2 + y^2 = 1 \}$ Se tiene entonces
  que la imagen de $f$ es $S'$
  Veamos que:
  \[
    f'(t) = (-\sin t, \cos t) \quad \forall \, t \in [0, 2 \pi ]
  \]
  Veamos que $f'(t) \neq 0$. Por la proposición 6, $f$ es rectificable y 
  \[
    L =
    \int_0^{2\pi} || f ' || = 2\pi.
  \]
\end{example}
\begin{example}
  \begin{figure}[ht]
    \centering
    \incfig{cicloide}
    \caption{cicloide}
    \label{fig:cicloide}
  \end{figure}
  Considere un ponto $P$ sobre una circunferencia de radio $a$. Si inicialmente
  $P$ coincide con el origen y la circunferencia rota sobre el eje $x$ en el
  sentido de las manecillas del reloj. $P$ sigue una trayectoria que se conoce
  como cicloide. Se puede verificar que dicha curva está dada por:
  \begin{align*}
    f: I &\to \mathbb{R}^2 \\
    \theta &\mapsto a(\theta - \sin \theta, 1 - \cos \theta
  \end{align*}
  Si $I \subset (0, 2\pi)$ entonces 
  \[
    f'(\theta) = a (1-\cos \theta, \sin \theta)
  \]
  Es una curva regular.\\
  Veamos también que $f$ definida en todo $[0, 2\pi]$ es rectificable. Por la
  proposición 6. Además
  \begin{align*}
    L= \int_0^{2\pi} ||f'||&= a\sqrt{2}\int_0^{2\pi} \big[ 1-\cos \theta \big] 
  \end{align*}
\end{example}
\begin{problem}
  Estudie la curva obtenida en forma análoga a la  anterior cuando el punto $P$
  pertenece al interior de la circunferencia.
\end{problem}
\begin{example}
  Considere la función:
  \begin{align*}
    f:[0,4\pi] &\to \mathbb{R}^3 \\
    t &\mapsto (a\cos t, a \sin t, t)
  \end{align*}
  Así:
  \[
    f'(t) = (-a \sin t, a \cos t, 1)
  \]
  Entonces $f$ es rectificable y:
  \begin{align*}
    L &= \int_0^{4\pi} ||f'|| \\
      &= \int_0^{4\pi} (a^2 + 1)^{\frac{1}{2}} \\
      &= 4\pi (a^2 + 1)^{\frac{1}{2}}
  \end{align*}
\end{example}
\begin{example}
  Considere una función $f: I \to \mathbb{R}$. tal que $f'$ está definida y es
  continua en $I$. Sea
  \begin{align*}
    F: I &\to \mathbb{R}^2 \\
    t &\mapsto (f, f(t))
  \end{align*}
  entonces $F$ es una curva regular ya que:
  \[
    F'(t) = (1,f'(t)) 
  \]
\end{example}
\begin{example}
  Considere la elipse $C$ cuya ecuación es $ \frac{x^2}{a^2} + \frac{y^2}{b^2}
  =1$. Así $C$ es la traza de la función:
  \begin{align*}
    f:[0, 2\pi] &\to  \mathbb{R}^2 \\
    t & \mapsto (a \sin t, b \cos t)
  \end{align*}
  Observe que $f'(t) \neq 0$.
\end{example}
\begin{problem}
  Encontrar una helice sobre la superficie de una esfera 
\end{problem}
16-ago-2019
\begin{definition}
  Sea $F: I \to \mathbb{R}^n$ una curva regular. Sean $C = f(I) y t \in I$. Se
  define el vector tangente unitario $C$ en $f(t)$, denotado por $T(t)$ como
  \[
    T(t) = \frac{f'(t)}{||f'(t)||}
  \]
  Considere la función 
  \begin{align*}
    s: I & \to \mathbb{R} \\
    t &\mapsto \int_a^t ||f'||
  \end{align*}
  Siendo $I=[a,b]$. $s(t)$ se denomina la longitud de arco de $f$ entre $f(a) y
  f(t)$. Observe lo siguiente: Sea $f: I \to \mathbb{R}^n$ una curva regular.
  Por el teorema fundamental del cálculo
  \[
    s' = ||f'||
  \]
  Se tiene de la definición del vector tangente que 
  \[
    T= \frac{f'}{s'} \quad \therefore \quad f' = s'T
  \]
  Si $f$ es de clase $C^2$ entonces se tiene:
  \begin{equation}\label{eq:2}
    f'' = s''T + s'T'
  \end{equation}
\end{definition}
\begin{lemma}
  Sea $g:I \to \mathbb{R}^n$ una función derivable. Si $||g||$ es constante,
  entonces $g(t)$ es ortogonal a $g'(t) \, \quad \forall \, t \in I$.
\end{lemma}
\begin{proof}
  Por hipótesis existe $c \in \mathbb{R}$ tal que $||g(t)|| =c \, \quad \forall t \in
  I$. Sea $t  \in I$ entonces:
  \begin{align*}
    c^2 = ||g(t)^2|| &= \langle g(t),g(t) \rangle \\
    0&= \langle g'(t), g(t) \rangle
  \end{align*}
  Si la función $T$ es derivable, entonces, dado que $||T(t)|| = 1$, se tiene
  del lema anterior que $T(t) \perp T'(t)$
\end{proof}
\begin{definition}
  Sea $f: I \mathbb{R}^n$ una curva regular. Sea $t \in I$ tal que $T'(t)$ está
  definida y es distinta de $0$. Se define entonces el vector normal unitario a
  $C= f(I)$ en $f(t)$ denotado por $N(t)$ como:
  \[
    N(t) = \frac{T'(t)}{||T'(t)||}
  \]
  Se tiene de lo anterior que $T(t)$ es ortogonal a $N(t)$. A la recta que pasa
  por $f(t)$ en la dirección del vector $N(t)$ se denomina recta normal
  principal.
\end{definition}
\begin{remark}
  De la identidad \eqref{eq:2} se tiene
  \begin{equation}\label{eq:3}
    f'' = s''T + s'N||T'|| 
  \end{equation}

\end{remark}
\begin{definition}
  Sea $f: I \to \mathbb{R}^3$ una curva regular de clase $C^2$. Sea $t \in I$
  tal que $T'(t) \neq 0$. El plano que pasa por $f(t)$, paralelo al plano
  determinado por $T(t)$ y $N(t)$, se denomina plano osculador a $C = f(I)$ en
  $f(t)$. El vector $T(t) \times N(t)$ se denomina vector binormal a $C$ en
  $f(t)$ y se denota por $B(t)$. Observe que $||B(t)||=1$.
\end{definition}
$T(t), N(t),B(t)$ son vectores unitarios, mutuamente ortogonales. 
$B(t)$ es un vector normal al plano osculador.
\begin{definition}
  Sea $f: I \to \mathbb{R}^3$ una curva regular. Sean $t_0, t_1 \in I$ 
  Se define la curvatura de $C=f(I)$ en $f(t_0)$ denotada por $\kappa(t_0)$ como
  \[
    \kappa = \lim_{t\to t_0} \frac{||T(t)- T(t_0)||}{|s(t)-s(t_0)|}
  \]
\end{definition}
\begin{figure}[ht]
  \centering
  \incfig{kappa}
  \caption{kappa}
  \label{fig:kappa}
\end{figure}
\begin{proposition}
  Sea $f:I \to \mathbb{R}^3$ una curva regula de clase $C^2$; sea $t_0 \in I$
  entonces $\kappa(t_0)$ está definida y es igual a:
  \[
    \frac{||T'(t_0)||}{||f'(t_0)||}
  \]
\end{proposition}
\begin{proof}
  \begin{align*}
    \lim_{t \to t_0} \frac{||T(t) - T(t_0)||}{|s(t)- s(t_0)|} &=
    \lim_{t \to t_0}\frac{\frac{||T(t)- T(t_0)||}{|t-t_0|}}{\frac{|s(t)-s(t_0)|}{|t-t_0|}} \\
    \frac{||T'(t_0)||}{|s'(t_0)|} &= \frac{||T'(t_0)||}{||f'(t_0)||}
  \end{align*} 
\end{proof}
\begin{definition}
  Sea $f: I \to \mathbb{R}^3$ una curva regular de clase $C^2$. si $t \in T$ y
  $\kappa(t) \neq 0$, se define el radio de curvatura de $C = f(I)$ en $f(t)$,
  denotado por $\mathfrak{c}(t)$ como $\frac{1}{\kappa(t)}$
\end{definition}
\begin{example}
  Considere la curva regular 
  \begin{align*}
    f: I &\to \mathbb{R}^2 \\
    t \mapsto (a\cos t, s \sin t).
  \end{align*}
  Veamos que:
  \begin{align*}
    f'(t) &= (- a \sin t, a \cos t) \\
    ||f'(t)|| &= a \\
    T(t) &= (- \sin t, \cos t) \\
    T'(t) &= (- \cos t, -\sin t) \\
  \end{align*}
  La proposición 7 implica que $\kappa(t) = \frac{1}{a}$. Entonces
  $\mathfrak{c}(t) = a$.

\end{example}
Observe que de la ecuación \eqref{eq:3} y la proposición 7, se tiene:
\begin{equation}\label{eq:4}
  f'' = s''T + N\kappa s'^2 
\end{equation}
\begin{proposition}
  Sea $g: I \to \mathbb{R}$ una función de clase $C^2$. Considere ahora la
  función 
  \begin{align*}
    f: I &\to \mathbb{R}^2\\
    t &\mapsto (t, g(t))
  \end{align*}
  La curvatura de $f$ en $t$ es 
  \[
    \kappa(t) =  \frac{|g''(t)|}{[1 + g'(t)^2]^{\frac{3}{2}}}
  \]
\end{proposition}
\begin{proof}
  $f$ es una curva regular de clase $C^2$. De la identidad \eqref{eq:4} se
  tiene:
  \begin{align*}
    \langle f'', T \rangle &= s''\langle T, T \rangle +
    \kappa(s')^2\langle N,T \rangle \\
                           &= s'' ||T||^2
  \end{align*}
  Como $||T||=1$ se tiene así $s'' \langle f'' , T \rangle$. Observe que
  $f' = (1, g')$, $f'' = (0, g'')$.
  \[
    T= \frac{f'}{||f'||} = \frac{(1,g')}{\big[ 1 + (g')^2\big]^{\frac{1}{2}}}
  \]
  así 
  \[
    s'' = \frac{g'g''}{\big[ 1 + (g')^2\big]^{\frac{1}{2}}}
  \]
  de \eqref{eq:4} se tiene
  \begin{equation}\label{prop:8:eq:1}
    \kappa N = \frac{f'' - s''T}{(s')^2 } 
  \end{equation}
  Observe que 
  \begin{align*}
    f'' - s''T &=  (0,g'') - \frac{g'g''}{\big[1 +(g')^2\big]^{1/2}}\frac{(1,g')}{\big[1 +
    (g')^2\big]^{1/2}} \\
               &= \frac{-g'g'', g''}{1 + (g')^2}
      \end{align*}
      de \eqref{prop:8:eq:1}:
      \[
        \kappa N = \frac{-g'g'', g''}{\big[1 + (g')^2\big]^2} 
      \]
      observe que $\kappa = ||\kappa N||$.
      Se tiene así finalmente que:
      \begin{align*}
        \kappa &=\frac{\big[ [(g')^2 +1](g'')^2\big]^{\frac{1}{2}}}{\big[1 +
        (g')^2\big]^{\frac{3}{2}}} \\
               &=  \frac{|g''(t)|}{[1 + g'(t)^2]^{\frac{3}{2}}} 
          \end{align*}
        \end{proof}
        \begin{proposition}
          Sea $f: I \to \mathbb{R}^3$ una curva regular de clase $C^2$. Entonces la
          curvatura está dada por la expresión
          \[
            \kappa = \frac{|| f' \times f'' ||}{||f'||^3}  
          \]
        \end{proposition}
        \begin{proof}
          De la identidad \eqref{eq:4} se tiene:
          \begin{align*}
            f' \times &= s'' (f' \times T) + \kappa (s')^2 (f' \times N) \\
            f' \times t &= 0 \\
            \mbox{así } f' \times f'' &=  \kappa (s')^2 (f' \times N) \\
            ||f' \times f'' || &= \kappa ||f'||^3 \\
            \kappa &=  \frac{||f' \times f''||}{||f'||^3}
          \end{align*}
        \end{proof}
        \begin{example}
          Considere la curva regular 
          \begin{align*}
            f:[0,2 \pi] &\to \mathbb{R}^3 \\
            t &\mapsto (a\sin t, b \cos t, 0)
          \end{align*}
          Veamos que:
          \begin{align*}
            f'(t) &= (a \cos t, - b \sin t, 0) \\
            ||f'|| &= (a^2 \cos^2 t + b^2 \sin^2)^{\frac{1}{2}}
          \end{align*}
          En consecuecia de la proposición anterior se tiene:
          \[
            \kappa = \frac{ab}{[a^2 \cos^2 t + b^2 \sin^2 t]^2}
          \]
          Así la curvatura es máxima ó mínima sí:
          \begin{align*}
            \kappa(0) = \kappa(\pi) &= \frac{b}{a^2} \\ 
            \kappa(\frac{\pi}{2}) = \kappa(\frac{3\pi}{2}) &= \frac{a}{b^2}
          \end{align*}
        \end{example}
21-ago-2019

Hemos definido a la curvatura de una curva en términos de la longitud de arco.
Si se definiera en términos del parámetro $t$, entonces sería dependiente de la
parametrización y no únicamente de la traza. \\ 

Sea $f: I \to \mathbb{R}^3$ una curva regular de clase $C^2$. Dado $\hat{t} \in
\kappa (\hat{t})$ la curvatura en $t$ es distinta de cero. Considere una
circunferencia cuyo radio sea $\rho(\hat{t})$, entonces su curvatura es
$\frac{1}{\rho(\hat{t})}$. Ésta circunferencia se denomina circunferencia
osculatriz de la traza de $f$ en $f(\hat{t})$.
Y es la curva que cumple que:
\begin{itemize}
  \item $f(\hat{t})$ está en la circunferencia.
  \item El centro de la circunferencia se encuentra en $f(\hat{t}) +
    \rho(\hat{t})N(\hat{t})$ 
\end{itemize}
\begin{problem}
Verifique que si $f: I \to \mathbb{R}$, es una función de clase $C^2$ lo mismo
que $g: I \to \mathbb{R}$, tales que para $t_0 \in I$, $f(t_0) = g(t_0)$,
$f'(t_0) = g'(t_0)$, $f''(t_0) = g''(t_0)$, entonces las curvas regulares
definidas por:
\begin{align*}
  F: I &\to \mathbb{R} \\
  t &\mapsto (t,f(t)) \\
  G: I &\to \mathbb{R} \\
  t &\mapsto (t,g(t)) \\
\end{align*}
Entonces la curvatura de $F(I)$ en $t_0$ es igual que la de $G(I)$ en $t_0$. y
determine las condiciones para las cuales la circunferencia osculatriz es la
misma.
\end{problem}
Sea $f: I \mathbb{3}^3$ una curva regular para la cual el vector binormal $B(t)$
está definido en todo $I$. Suponga que la función $B$ es derivable. Entonces se
tiene lo siguiente:
\begin{align*}
  B' \times N = (T \times N)' \times N &= \big[(T' \times N ) + (T \times N')
  \big] \times N \\ 
                                       &= 0
\end{align*}
\begin{definition}
 Como los vectores $B'$ y $N$ son colineales, tenemos que  $\forall \, t \in I$,
 existe un escalar $\tau(t)$ tal que
 \[
   \frac{B'(t)}{s'(t)} = \tau(t) N (t)
 \]
 A $\tau(t)$ se denomina la torsión de la traza de $f$ en $f(t)$.
\end{definition}
\section{Ecuaciones de Frenet-Serret}
\begin{proposition}
 Sea $f$ una curva regular de clase $C^3$. Entonces:
 \begin{enumerate}
   \item $T' = \kappa s' N$ \\
   \item $N' = -\kappa s' T - \tau s'B$
 \end{enumerate}
\end{proposition}
\begin{proof}
 Observe que $T' = N ||T'||$. Como $||T'|| = \kappa s'$, entonces tenemos 
 \[
   R' =\kappa s' N
 \]
 De la definición de torsión se tiene
 \[
   B' = \tau s' N
 \]
 Así mismo
 \begin{align*}
   B' &= (T \times N)' \\
      &= (T \times N') + (T' \times N ) \\
      &= T \times N'
 \end{align*}
 Así $\tau s' N = T \times N'$. Así
 \begin{align*}
   \tau s' N &=  
 \end{align*}
\end{proof}
Las ecuaciones
\begin{align*}
  T'&= \kappa s' N \\
 N' &= -\kappa s' T - \tau s' B \\
 B' &=\tau s' N
\end{align*}
Se conocen como ecuaciones Frenet-Serret. Fueron obtenidas en 1847 por J. Frenet

\begin{proposition}
 Considere una curva regular $f$ de clase $C^3$. Sea $t \in I$ tal que $\kappa
 \neq 0$. entonces la torsión
 \[
   \tau (t) = -\frac{\langle f'(t) \times f'' (t), f'''(t)\rangle}{||f'(t) \times
   f''(t) ||^2}
 \]
\end{proposition}
\begin{proof}
  De la identidad \eqref{eq:4}.
  se tiene
  \[
    f''' = s'''T +s''T' + \kappa'Ns^2' +  \kappa [2s's''N + (s')^2N']
  \]
  Haciendo uso de las dos primeras ecuaciones de Frenet se tiene
  \begin{align*}
    f''' = s'''T + s''s'\kappa N &+ \kappa N s^2' + 2\kappa  s's''N \\
                                 &+ \kappa(s'^2) [-\kappas' T - \tau s' B]
  \end{align*}
así:
\begin{align}\label{prop:11:lambda}
  f''' = [s'''-\kappa^2(s')^3]T &+ [3 \kappa s's'' + \kappa' (s')^2] N \\
                                &-\kappa \tau (s')^3B
\end{align}
De \eqref{eq:4} se tiene
\begin{align*}
  f\times f'' &= s's''(T \times T) + \kappa (s')^3 B \\
              &= \kappa (s')^3 B \\
  ||f \times f''|| &= \kappa(s')^3
\end{align*}
Como $\langle B, T \rangle = \langle B, N \rangle =0$ se tiene de lo anterior
\begin{align*}
  \langle f'\times f'' ,  f''' \rangle &= - \tau [\kappa(s')^3]^2 \\
                                       &= -\tau ||f' \times f''||^{2}
\end{align*}
Así si $\kappa \neq 0$ se concluye
 \[
   \tau (t) = -\frac{\langle f'(t) \times f'' (t), f'''(t)\rangle}{||f'(t) \times
   f''(t) ||^2}
 \]
\end{proof}
